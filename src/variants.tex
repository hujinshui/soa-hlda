\section{Variants of the HLDA}
\label{sec:variants}

There is several variants of the HLDA. We present them briefly.

First we will present a method which is a combination of the LDA and
the HDLA. In a second time, we present a method which allows to make our
computation faster.

\subsection{Smooth HLDA}
\label{sec:smooth-hlda}

If the HLDA works on a lot of classes, the estimation of the class
covariance becomes noisy. The LDA doesn't have this problem because
the within-class covariance matrix is the same for all the classes.

So, what is the solution when there is a lot of classes and the data
are not homoscedastics ? Burget\cite{burget.2004} presents a smooth
LDA. It is a combination of the LDA and the HLDA and it tries to get
the advantages of the two methods.

In this, we estimate the class covariance matrix as:

$$\breve\Sigma^{\left( j\right) }=\propto \hat\Sigma^{\left( j\right) } +\left( 1- \propto\right) \Sigma_{wc}$$

Where $\breve\Sigma^{\left( j\right) }$ is the smoothed estimated
covariance for the class $j$ used by the HLDA.
$\hat\Sigma^{\left(j\right) }$ is the covariance computed by the HLDA, and
$\Sigma_{wc}$ is the one computed by the LDA.

So it is clear that if $\propto$ is 1, the SHLDA is a HLDA, and if it
is equal to 1 it is a LDA.

\subsection{Diagonal Block HLDA}
\label{sec:db-hlda}

The HLDA is a very time consuming method. Performing it on real data set can be impossible as the data number of dimensions can be very high.
So an extension of the HLDA is to divide the research of $\hat{A}$ into independant subproblems.
This method is called Diagonal Block HLDA\cite{zhang2011}
$$
\hat{A} = \left [
\begin{array}{ccc}
\hat{A}^{1} \\
& \ddots \\
& & \hat{A}^{N}
\end{array}
\right ]
$$

With $\hat{A}^l, \; l = 1...N$ a subproblem to solve.
and $\hat{A}^l = argmax_{A_l}L^l (A^l; \{x_i\}^l)$

The equation for $L$ is:
$$
\begin{array}{cl}
  L(A; \{x_i\}) & = \sum\limits_{j = 1}^J \frac{I_j}{2} \sum\limits_{n = 1}^N Log \left ( \frac{|A^n|^2}{|Diag(A_{U^n}^n W_j^n (A_{U^n}^n)^T)||Diag(A_{M - U^n}^n T^n (A_{M - U^n}^n)^T)|} \right ) \\
  & = \sum\limits_{n = 1}^N L^n ( A^n; \{x_i^n\})
\end{array} 
$$

Each problem can be solved independantly.

%%% Local Variables:
%%% mode: latex
%%% TeX-master: "../hlda"
%%% End:
