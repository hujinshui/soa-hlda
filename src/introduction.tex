% *** Pourquoi la réduction de dimension, et la décorellation des données.
% Cours Réda
% *** Pourquoi la LDA est insuffisante.
% Histoire des matrices de covariance.
% *** En quoi la HLDA résout le problème
% Ce qu'elle permet

\section{Introduction}
\label{sec:introduction}

In this state of the art, we present a dimension reduction method
which is a generalization of the Linear Discriminant Analysis(LDA).
One main flaws of the LDA method
is that it discriminates classes only by their mean and do not take into
account their variances. The Heteroscedastic Linear Discriminant Analysis (HLDA)
method relaxes this constraint to the price of a more complex solution
(not analytical anymore). Throughout the sections, the differences between LDA and
HLDA are emphasized.

This document is structured as follow: the first part quickly review the LDA method,
then the second part describe the HLDA method. In the third part some benchemarks
are presented. Finally the fourth part presents some variants of the HLDA and concludes.
