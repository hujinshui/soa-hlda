\subsection{HLDA}

\subsubsection{Differences between LDA and HLDA}

HLDA (Heteroscedastic Linear Discriminant Analysis) is another method to find a linear transformation mapping the problem space into a smaller one.

The HLDA finds the projection matrix A such that:
$$y = Ax$$
with y a $p$ dimensions variable and $p < n$.\\

The purpose of the HLDA method is to release the homoscedasticiy constraint of the LDA, allowing different variances for each classes. The two assumptions made by the method are:

\begin{enumerate}
  \item The class distributions are $n$-dimensional Gaussian distributions.
  \item This space of size $n$ can be split into two subspaces of size $p$
    and $n - p$ (with $p < n$). The first one is different for all classes and contains meaningful information while the second one is the same for the whole dataset and contains only noise.
\end{enumerate}

The projection matrix A can then be rewritten following those assumptions:
$$A = \left [
  \begin{array}{c}
    A_p\\
    A_{n-p}
  \end{array}
\right ]
$$

it is also possible to find an expression for the mean and variance of each classes:
$$\mu_j =
\left [
  \begin{array}{c}
    \mu_{j,p}\\
    \mu_{0,(n-p)}
  \end{array}
\right ]
$$

$$\Sigma_j =
\left [
  \begin{array}{cc}
    \Sigma_j^p & 0\\
    0 & \Sigma_0^{(n-p)}\\
  \end{array}
\right ]
$$
Where $\mu_0$ and $\Sigma_0$ are supposed to be common to all classes.

Based on those definitions, it is possible to define a model representing the new dataset.

\subsubsection{HLDA Gaussian Model}

The Gaussian probability density function for this model is:
$$P(y_i) = \frac{\mid \theta \mid}{\sqrt{(2\pi)^n \mid \Sigma_{g(i)} \mid}}
e^{-\frac{1}{2} (y_i - \mu_{g(i)})^t \Sigma_{g(i)}^{-1} (y_i - \mu_{g(i)})}$$
With $g(i)$ the function associating each sample with its label.

The variables of the model are:
\begin{itemize}
  \item $A$: the projection matrix
  \item $\mu_j$ the mean of class $j$
  \item $\Sigma_j$ the covariance of class $j$
\end{itemize}

Note that $\mu_j$ and $\Sigma_j$ can depend on $A$ thus no analytical solution can be found (in contrast to the LDA method). In order to find estimators for those parameters Kumar\cite{kumar.1997} uses the maximum log-likelihood method and find two different solutions depending on the type of $\Sigma_j^p$.
To estimate the parameters found, a numerical-optimisation method is required.

\subsubsection{Full-rank $\Sigma_j^p$}

The Log-likelihood of the data $L_F = \sum\limits_{i = 1}^N log P(x_i)$ under the linear transformation $A$ and under the constrained Gaussian model assumption for each class is:
$$\begin{array}{cl}
  Log L_F(\mu_j, \Sigma_j, A; \{x_i\}) = & - \frac{1}{2} \sum\limits_{i = 1}^N\{(A^T x_i - \mu_{g(i)})^T\Sigma_{g(i)}^{-1}
  (A^T x_i - \mu_{g(i)}) \\
  & + Log((2\pi)^n|\Sigma_{g(i)}|)\} + Log|A|
\end{array}$$

Maximizing the Log-likelihood, knowing that there is a lot of parameters is very time consuming.
First, the values of the mean and variance parameters that maximize the log-likelihood with a fixed linear transformation $A$ are estimated.
This estimations are found by calculating the partial derivation of the log-likelihood equation. The maximum estimated values are the point where the partial derivatives are zero. \\
The mean and variance estimates are:
$$
\begin{array}{cc}
  \frac{\partial Log L_F(\mu_j, \Sigma_j, A; \{x_i\})}{\partial \mu_j} \Rightarrow \left \{
  \begin{array}{ccc}
    \hat\mu_j^p & = & A_p^T\bar{X_j} ,\; j = 1...J\\
    \hat\mu_0 & = & A_{n - p}^T\bar{X}
  \end{array}
  \right . \\
  \frac{\partial Log L_F(\mu_j, \Sigma_j, A; \{x_i\})}{\partial \Sigma_j} \Rightarrow \left \{
  \begin{array}{ccc}
    \hat\Sigma_j^p & = & A_p^T\bar{W_j}A_p, \; j = 1...J\\
    \hat\Sigma^{n - p} & = & A_{n - p}^T\bar{T}A_{n - p}
  \end{array}
  \right .
\end{array}$$

The substitution of this estimated values in the Log-likelihood equation gives:
$$
\begin{array}{cl}
  Log L_F(A; \{x_i\}) = & \frac{-N n}{2}Log(2\pi)-\frac{N}{2}Log|(A_{n - p}^T\bar{T}A_{n - p})| \\
  & - \sum\limits_{j = 1}^J
  \frac{N_j}{2}Log|(A_p^T\bar{W_j}A_p)| \\
  & - \frac{1}{2}\sum\limits_{j = 1}^J\sum\limits_{g(i) = j} ((x_i - \bar{X_j})^TA_p(A_p^T\bar{W_j}A_p)^{-1}
  A_p^T(x_i - \bar{X_j})) \\
  & - \frac{1}{2}\sum\limits_{j = 1}^J\sum\limits_{g(i) = j} ((x_i - \bar{X})^TA_{n - p}(A_{n - p}^T\bar{T}A_{n - p})^{-1}
  A_{n - p}^T(x_i - \bar{X})) + N Log|A|
  \end{array}$$

The simplification of $L_F(A, \{x_i\})$ and it's maximization gives:
$$ \hat{A_F} = argmax_A\{-\frac{N}{2}Log|(A_{n - p}^T\bar{T}A_{n - p})| - \sum\limits_{j = 1}^J\frac{N_j}{2}Log|(A_p^T\bar{W_j}A_p)|
+ N Log|A|\}$$

$\hat{A_F}$ can be computed by a numerical-optimisation method.
To perform a dimension reduction transformation, we use only the first p columns of $\hat{A_F}$.

\subsubsection{Diagonal $\Sigma_j^p$}
Assuming the within-class covariance matrix to be diagonal, the calculus becomes easier.
From now, the matrix $\Sigma_j$ are supposed to be diagonal and their diagonal elements will be called $\{\sigma_j^1...\sigma_j^p,
\sigma^{p + 1}...\sigma^n\}$
Note that when all $\Sigma_j$ are equals, the result is the same than the LDA one.

In this case, the log-likelihood of the data can be written as:
$$
\begin{array}{cl}
  Log L_D(\mu_j, \Sigma_j, A; \{x_i\}) = & -\frac{N n}{2} Log (2\pi) + N Log|A| - \frac{N}{2}\sum\limits_{k = p + 1}^n Log|\sigma^k| \\
  & - \sum\limits_{j = 1}^J \frac{N_j}{2} \sum\limits_{k = 1}^p Log|\sigma_j^k| - \frac{1}{2}\sum\limits_{j = 1}^J\sum\limits_{g(i) = j}
  \sum\limits_{k = 1}^p \frac{(\vec{A_k^T} - \mu_{j,k})^2}{\sigma_j^k} \\
  & + \frac{1}{2}\sum\limits_{j = 1}^J\sum\limits_{g(i) = j}
  \sum\limits_{k = p + 1}^p \frac{(\vec{A_k^T} - \mu_{0,k})^2}{\sigma^k}
\end{array}
$$

Using the same method than before to maximize $\mu_j$ and $\Sigma_j$

$$
\begin{array}{cc}
  \frac{\partial Log L_D(\mu_j, \Sigma_j, A; \{x_i\})}{\partial \mu_j} & \Rightarrow \left \{
  \begin{array}{ccc}
    \hat\mu_j^p & = & A_p^T\bar{X_j} ,\; j = 1...J\\
    \hat\mu_0 & = & A_{n - p}^T\bar{X}
  \end{array}
  \right . \\
  \frac{\partial Log L_D(\mu_j, \Sigma_j, A; \{x_i\})}{\partial \Sigma_j} & \Rightarrow \left \{
  \begin{array}{ccc}
    \hat\Sigma_j^p & = & Diag(A_p^T\bar{W_j}A_p), \; j = 1...J\\
    \hat\Sigma^{n - p} & = & Diag(A_{n - p}^T\bar{T}A_{n - p})
  \end{array}
  \right .
\end{array}
$$

After substitution in the main equation:
$$
\begin{array}{cl}
  Log L_D(A; \{x_i\}) = & - \frac{N n}{2} Log(2\pi) - \frac{N}{2} Log|Diag(A_{n - p}^T\bar{T}A_{n - p})| \\
  & - \sum\limits_{j = 1}{J}\frac{N_j}{2} Log|Diag(A_p^T\bar{W_j}A_p)| \\
  & - \frac{1}{2} \sum\limits_{j = 1}^J \sum\limits_{g(i) = j} ((x_i - \bar{X} )^T A_{n - p} Diag( A_{n - p}^T \bar{T} A_{n - p} )^-1 A_p^T(x_i - \bar{X}))\\
  & + N Log |A|
\end{array}
$$

After the simplification of $Log L_D(A; \{x_i\})$ and it's maximisation:
$$
\hat{A_D} = argmax_A\{-\frac{N}{2} Log|Diag(A_{n - p}^T \bar{T} A_{n - p})| - \sum\limits_{j = 1}^J \frac{N_j}{2} Log|Diag(A_{n - p}^T \bar{W_j} A_{n - p})| + N Log|A|\}
$$

$\hat{A_D}$ can be computed by a numerical-optimisation method.
To perform a dimension reduction transformation, we use only the first p columns of $\hat{A_D}$.

\subsubsection{Computing $\hat A$}

Now that the equations of $\hat{A_F}$ and $\hat{A_D}$ are defined, their values are calculated by numerical-optimization methods.
All methods that allow to find a maximum/minimum values in a function could work.
Kumar\cite{kumar.1997} uses a Gradient descent algorithm to perform the calculation.
